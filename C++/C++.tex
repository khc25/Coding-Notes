\documentclass[a4paper]{article}

\def\npart{I}
\def\nterm {Sem 2}
\def\nyear {2017-2018}
\def\nlecturer {Brian}
\def\ncourse {C++}

\input{header}

\usepackage{listings}


\lstset{language=C++,
                basicstyle=\ttfamily,
                keywordstyle=\color{blue}\ttfamily,
                stringstyle=\color{red}\ttfamily,
                commentstyle=\color{green}\ttfamily,
                morecomment=[l][\color{magenta}]{\#}
}


\begin{document}

\maketitle


\tableofcontents

\section{Print out the code}
\subsection{Hello world!}
\begin{lstlisting}
    #include<stdio.h>
    #include<iostream>
    // A comment
    int main(void)
    {
    printf("Hello World\n");
    return 0;
    }
\end{lstlisting}



\section{Template}
\begin{lstlisting}
 #include <iostream>
 using namespace std;
 int main(){
           float hkd;
           float result;
           cin >>hk;
           result=hkd*14.2
           cout <<result; // print out the result
 }

\end{lstlisting}



\section{Data type}
\subsection{Variable and Constant}

\subsubsection{Numerical}
\begin{itemize}
\item int: Integer type
\item flaot, double: 
\end{itemize}

\begin{remark}
How is Float works:\\
\\
\begin{tabular}{|c|c|c|c|c|c|c|c|}
\hline
 $\frac{1}{2}$&$ \frac{1}{2}^2$&$\frac{1}{2}^3 $&$\frac{1}{2}^4 $&$ \frac{1}{2}^5 $&$\frac{1}{2}^6 $&$\frac{1}{2}^7 $&$\frac{1}{2}^8$\\
\hline
\end{tabular}
\\
How is Double work:\\
\\
\begin{tabular}{|c|c|c|c|c|c|c|c|c|c|c|c|c|c|c|c|}
\hline
 $\frac{1}{2}$&$ \frac{1}{2}^2$&$\frac{1}{2}^3 $&$\frac{1}{2}^4 $&$ \frac{1}{2}^5 $&$\frac{1}{2}^6 $&$\frac{1}{2}^7 $&$\frac{1}{2}^8$&$\frac{1}{2}^9$&$ \frac{1}{2}^{10}$&$\frac{1}{2}^{11} $&$\frac{1}{2}^{12}$ & $\frac{1}{2}^{13}$ &$\frac{1}{2}^{14} $&$\frac{1}{2}^{15} $&$\frac{1}{2}^{16}$\\
\hline
\end{tabular}
\end{remark}
\begin{eg}


\end{eg}

\subsubsection{Character}

\begin{itemize}
\item char:
\end{itemize}

\subsubsection{Logic}

\begin{itemize}
\item bool:Boolean (true, false)
\end{itemize}

\subsubsection{Other}

\begin{itemize}
\item void:
\end{itemize}


\section{Basic Operators}
\subsection{Type of Operators}

\subsubsection{Number Operator}


\subsubsection{Comparative Operator}
This operators structure will return to boolean
\begin{lstlisting}
\\Equality operators
== // equal to
!= // not equal to

\\Relational operators
>  // greater than
>=  // greater and equal than
<  // 
<=  //
\end{lstlisting}

\subsubsection{Logical Operator}

This operators structure will return to boolean
\begin{lstlisting}
!  // not
&&  // and
||  // or
\end{lstlisting}

\begin{eg}
Given integer variable i,j and k, what are the outputs when running the program fragment below?
\begin{lstlisting}
k = (i=2) \&\& (j=2) ;
cout << i << j << endl; /* 2 2 */

k = (i=0) \&\& (j=3) ;
cout << i << j << endl; /* 0 2 */

k = i ||  (j=4) ;
cout << i << j << endl; /* 0 4 */

k = (i=2) || (j=5) ;
cout << i << j << endl; /* 2 4 */

\end{lstlisting}
Answers:

\end{eg}



\subsubsection{Conditional Operator}

\subsubsection{Comma operator}






\section{Method}


\section{Flow control}
\subsection{If else statement}
if must in first part
else if
else must in last part

\begin{lstlisting}
if (logical_expression){

    statement;
    statement;
    
}
else if (logical_expression){

     statement;
     statement;

}

else{

   statement;
   statement;
   
}
\end{lstlisting}

\subsection{Switch statement}
What is Switch statement look like:
\begin{lstlisting}
switch (expression) {
        case constant-expr1: statement1
        case constant-expr2: statement2
        ...
        ...
        case constant-exprN: statmentN
        default: statement
}

\end{lstlisting}

\begin{eg}
Here is the example for using switch statement:
\begin{lstlisting}
while ((c = getchar()) != EOF) {  /* get a char */
      switch (c) {
        case ?0?: case ?1?: case ?2?: case ?3?: case ?4?:
        case ?5?: case ?6?: case ?7?: case ?8?: case ?9?:
         digit_count++;   /* no braces is needed */
         break;
        case ? ?: case ?\n?: case ?\t?:
         white_character_count++;
         break;
        default:
         other_character_count++;
         break; 
   }
}
\end{lstlisting}
\end{eg}

\section{Loop}

\subsection{While Loop}

\subsection{Do-while Loop}

\subsection{For Loop}

\section{Functions}
\subsection{Introduction}
\begin{eg}

\end{eg}
\begin{lstlisting}
 #include <iostream> using namespace std;
void printHello(int n){
  for (int i=0;i<n;i++)
      cout <<"Hello" <<endl;
}
void main() {
  printHello(10);
}
\end{lstlisting}

\subsection{Calling functions}

\begin{eg}

\end{eg}
\begin{lstlisting}
 #include <iostream> using namespace std;
void printHello(int n){
  for (int i=0;i<n;i++)
      cout <<"Hello" <<endl;
}
 
 void main() {
int x=1;
  printHello(x);
  printHello(x+3);
  printHello(10);
}
\end{lstlisting}

\subsection{Multiple parameters}

\begin{eg}

\end{eg}

\begin{lstlisting}
#include <iostream> 
using namespace std;
int maxValue(int a, int b){
      int m=a;
      if (b>a)
               m=b;
      return m;
}
 
void main() {
       int x , y=4, z=1;
       x = maxValue(4,2);
}
\end{lstlisting}

\section{Global and local variable}
\subsection{Introduction}

\subsection{Example}
\begin{eg}

\end{eg}

\begin{lstlisting}
#include <iostream> 
using namespace std; 
int num1=4;
int num2=9;
int maxValue(int a, int b){ /*  Local(maxValue)  */
       int m=a;
       if (b>a) 
       m=b;  /*  Local(maxValue)  */
       return m; 
}

void main() {
    int x;  /* Local(main) */
    x = maxValue(num1,num2); /* Global   */
}

\end{lstlisting}

\subsection{Parameters Passing: Pass-by-value}
\begin{itemize}
\item When a function is invoked, the arguments within the parentheses are passed using a pass-by-value
\item Each argument is evaluated, and its value is used locally in place of the corresponding formal parameter.
\end{itemize}

\begin{eg}

\end{eg}

\begin{lstlisting}
void f (int x) {
       x=4; //we modify the value x to 4
       //Do we modify y at the same time? NO
       y=4; //syntax error: y is local to main
}

void main () {
       int y=3;
       f(y);
       cout << y; //print 3, y remains unchanged
}
\end{lstlisting}

\begin{itemize}
\item
\item
\item
\item
\end{itemize}


\section{Class}
\subsection{Constructor}
\subsection{this}















\end{document}